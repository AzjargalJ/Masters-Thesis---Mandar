\chapter{Introduction}
\label{cha:intro}
The $21^{st}$ century stands out in how mankind learned the value of storing and making predictions/decisions from large volumes of data. A significant aspect of large scale data analysis is distributed computation frameworks like \textit{High Performance Computing}, \textit{Message Passing Interface} etc. Recently large scale commodity hardware clusters have replaced the two former frameworks as the most popular model for parallel data analysis. With this crucial change in hardware came a change in computational models as well. It is at this juncture that distributed \textit{Map Reduce} became the de-facto computational philosophy for large scale data analysis and  words such as \textit{Hadoop} \cite{Hadoop:2005}, \cite{chang2008bigtable}, \cite{Borthakur2011} and \textit{Apache Spark} \cite{Zaharia2010}, \cite{Spark:2010} have become synonymous with large scale data analysis and machine learning.

Along with innovation in hardware design and distributed computing models, there came a need for good programming libraries and frameworks to work with various Machine Learning models on large data sets. It was demonstrated in \cite{10.1109/MIS.2009.36} that a gigantic language corpus encapsulates almost all aspects of human language and speech. So far the prevalent `motto' in the Internet industry has been ``large data, simple models". Often, this is misunderstood as the Machine Learning translation of \textit{Occam's Razor}. The bias-variance trade-off \cite{Valentini2004} is a far better mechanism to ensure the model does not become overly complex, and this, rather than restricting the user to simple models, is the real Occam's razor in training a model. 

Therefore, in order to extract maximum value from large scale data, it is important to have the flexibility to train and compare different model families before arriving at the one that fits the requirement of the user. Therefore one must be able to train general nonlinear models and tweak them by changing the various components which they employ to learn (i.e., a model may be linear or kernel based, it can be optimized by various methods like \textit{Stochastic Gradient Descent}, \textit{Conjugate Gradient}, etc.). This is not possible in a rigid, monolithic programming framework. Modularity, extensibility and ease of usage are of paramount importance while designing Machine Learning software for large scale data applications.

Scala \cite{scala-overview-tech-report} a multi-paradigm Java Virtual Machine (JVM) based programming language has gained popularity for its expressiveness and performance. It's power and easy interpolability with Java has quickly made it the language of choice for production grade Machine Learning software development. The current state of the art in distributed Machine Learning in Scala is the \textit{MLLib} module in \textit{Apache Spark} \cite{Meng}. It has implementations of Linear SVM and Logistic Regression for solving binary classification problems. But a crucial component missing in \textit{MLLib} and all distributed Machine Learning libraries is the ability to learn classification models with nonlinear decision boundaries. FS-Scala aims to solve the problem of scalable non-linear classification models by implementing the \textit{Fixed-Size Least Squares Support Vector Machine} (FS-LSSVM) algorithm \cite{DeBrabanter2010,Suykens2002} with model tuning capabilities.

In recent literature we find sparse reductions to FS-LSSVM methods \cite{Mall2015,Mall2013}. The authors in \cite{Mall2015,Mall2013} explored the sparsity vs error trade-off for FS-LSSVM models. Even though they run experiments on large scale datasets like Forest Cover dataset, the scalability of these methods are restricted to available memory on a single machine. Moreover, they don't exploit the possibility of parallelism available in several components of the FS-LSSVM model. Another work \cite{Mall2014} converts the Big Data into a Big Network and then uses a network based subset selection technique (\textit{Fast and Unique Representative Subset selection} (FURS) \cite{Mall2013FURS}) to obtain a representative subset of the original data. It then builds a FS-LSSVM model using this subset. However, in this thesis we showcase that we can parallelize the subset selection technique which maximizes the \textit{Quadratic R\`enyi Entropy} for Big datasets and use the generated subset as the set of prototype vectors (PV) essential for building the FS-LSSVM model.

%%% Local Variables: 
%%% mode: latex
%%% TeX-master: "thesis"
%%% End: 
